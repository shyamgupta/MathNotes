\documentclass{book}
\raggedbottom
\usepackage[a4paper,margin=1in]{geometry}
\usepackage{wrapfig}
\usepackage{mdframed}
\usepackage{soul}
\usepackage{gensymb}
\usepackage{graphicx}
\usepackage[export]{adjustbox}
\graphicspath{{./images/}}
\usepackage{datetime}
\newdate{date}{02}{04}{2020}
\title{Mathematics Notes}
\author{Shyam S. Gupta}
\date{\displaydate{date}}
\usepackage{hyperref}
\hypersetup{          
	pdfpagemode=UseOutlines,    % this is the option you were lookin for
}
\begin{document}
	\maketitle
	\tableofcontents

	\chapter{Number Systems}
	\section{Introduction}
	\begin{itemize}
	\item \textbf{N, Natural Numbers}: 1,2,3...
	\item \textbf{W, Whole Numbers}: 0,1,2,3,...
	\item \textbf{Z, Integers}: ...-3,-2,-1,0,1,2,3,...
	\item \textbf{Q, Rational Numbers}
	\item \textbf{P, Irrational Numbers}
	\item \textbf{R (Real Numbers)}: Collection of all rational and irrational numbers, i.e. a real number is either rational or irrational.
	\end{itemize}
	
	\section{Real Number Line}
	\paragraph{Number Line}
	On the number line, distances from a fixed point are marked in equal units positively in one direction and negatively in the other . The point from which the distances are marked is called the \textbf{origin}. We use the number line to represent the numbers by marking points on a line at equal distances. If one unit distance represents the number 1, then 3 units distance represents the number 3, 0 being at the origin. The point in the positive direction at a distance $r$  from the origin represents the number r . The point in the negative direction at a distance r  from the origin represents the number $-r$.
	\paragraph{Real Number Line}
	Every real number is represented by a unique point on the number line. Also, every point on the number line represents a unique real number - this is why the number line is called the Real Number Line.
	
	\includegraphics[scale=0.7]{numbersystem}
	
	\section{Rational Numbers}
	\textbf{Definition: }A number 'r' is called a rational number if it can be written in the form $p/q$, where:
	\begin{itemize}
	\item 	$p$ and $q$ are integers 
	\item $q \neq 0$
	\item $p$,$q$ have no common factors other than 1 (i.e. p and q are co-prime).
	\end{itemize}
	
	\underline{	\textbf{Important properties of Rational Numbers:}}
	\begin{itemize}
	\item An integer, for example $25$, can be written as $25/1$, therefore rational numbers include natural numbers, whole numbers and integers.
	\item Rational numbers do \textbf{NOT} have a unique representation in the form $p/q$. For example, $1/2 = 2/4 = 10/20$ and so on. These are \textbf{equivalent rational numbers}.  When we say $p/q$ is a rational number or represent it on the number line, we assume that $p$ and $q$ are co-prime. On the number line, among the infinitely many fractions equivalent to $1/2$, we will choose $1/2$ to represent them all.
	\item There are infinitely many rational numbers between any two given rational numbers.
	\item $0$ is a rational number
	\item Decimal expansion of a rational number is either terminating (example $1/4 = 0.25$) or non-terminating recurring (example $1/3 = 0.333.....$)
	\item Number of recurring entries is less than the divisor. For example, in $1/7$, there are six entries $326451$ in the repeating string of remainders and $7$ is the divisor. If the remainder's repeat, then we get a repeating block of digits in the quotient ($142857$ in case of $1/7$).
	\item If $x$ is a rational number (of the form $p/q$, where $p$ and $q$ are co-prime), whose decimal expansion terminates,  then prime factorization of $q$ is of the form $2^m5^n$, where $n$ and $m$ are non-negative numbers. Vice versa is also true, i.e. if prime factorization of  $q$ sis of the form $2^m5^n$, then $x$ has a decimal expansion which terminates. If the prime factorization of $q$  is not of the form $2^m5^n$ then the decimal expansion is non-terminating and recurring.
	\end{itemize}
	
	\section{Irrational Numbers}
	\begin{itemize}
	\item A number $s$ is called irrational, if it \textbf{cannot} be written in the form $p/q$, where p and q are integers and $q \neq 0$
	\item The decimal expansion of an irrational number is non-terminating and non-recurring.
	\end{itemize}
	
	\section{Examples}
	\begin{mdframed}[backgroundcolor=yellow]
		\textbf{Example: Express $0.333..$ as a rational number in the $p/q$ form.}
	\end{mdframed}
	\textbf{Solution:} \\
	$x = 0.333.....$ \\
	$10x = 3.333.... = 3 + 0.333... = 3 + x$ $=>$ $9x =3$ $=>$ $x=1/3$ \\
	
	\begin{mdframed}[backgroundcolor=yellow]
		\textbf{Example: Locate $\sqrt{3}$ on the number line}
	\end{mdframed}
	\includegraphics[scale=0.7]{squareroot} \\ \\
\textbf{Solution} \\ \\
Draw triangle $BCD$, where $BC=CD=1 => BD = \sqrt{2}$ \\
Draw $BY$ perpendicular to $DB$ , therefore $DY = \sqrt{3}$ \\
With center $D$ and radius $DY$ draw an arc intersecting the number line to locate $\sqrt{3}$ \\
Using this approach, you can locate the square root of other integers.
	\section{Square Root of Real Numbers}
If $a$ is a natural number, then $\sqrt{a} = b$ means $b^2 = a$ and $b>0$. In general, if $a>0$ be a real number and $n$ is a positive integer, then $\sqrt[n]{a} = b$ if $b^n =a$ and $b>0$. We follow the below convention:
\begin{mdframed} [backgroundcolor=yellow] 
	$\sqrt x$  represents the \textbf{positive square root} of $x$ and is called the  \textbf{principle square root } of $x$.
\end{mdframed}
For \textbf{$y=\sqrt[n]{x}$}:
\begin{table}[ht]

	\begin{tabular}{c c c c }
		\hline
		x & n & Result & Example \\
		\hline
		$x<0$ & $odd$ & There exists one negative $n^{th}$ root of $x$& $-2$ is the principle cube root of $-8$ \\
		$x<0$ & $even$ & $n^{th}$ root of $x$ is not a real number & $\sqrt{-16}$ is not Real
	\end{tabular}
\end{table}
\\
We must emphasize that \hl{$\sqrt{x^2} = \mid x \mid \neq x$}. For example, $\sqrt{-3^2} = \sqrt{9} = 3$ which is the absolute value of $\mid -3 \mid$.

	\section{Absolute Value}
	We now introduce the geometric concept of distance in the real number system. Let $a$ be the coordinate of a point on the number line - the number of units between this point and the origin $0$ is called the absolute value of $a$ and is denoted by \textbf{$\mid a \mid$} . Thus the absolute value of a number \textbf{gives only its distance from the origin and not its direction.}
	
	\includegraphics[scale=0.6]{absvalue} \\


In general,  if $a$ is any real number, then $\mid a \mid$ = $\mid -a \mid$, therefore the absolute value of a number is never negative. \\
Concept of absolute value can be used to measure the distance between two points - if $x_1$ and $x_2$ are coordinates for two points $A$ and $B$ on the real number line, then the distance between $A$ and $B$ is $\mid x_1 - x_2 \mid$ \\

\underline{\textbf{Important Propositions related to Absolute Value:}} \\

\begin{itemize}
	\item $\mid a \mid = \mid -a \mid$
	\item $\mid a.b \mid = \mid a \mid.\mid b \mid$
	\item $ab \leq \mid ab \mid$
	\item $\mid a+b \mid \leq \mid a \mid + \mid b \mid$
	\item $\mid a \mid - \mid b \mid \leq \mid a-b \mid$
	\item $\mid a/b \mid = \mid a \mid / \mid b \mid$
	\item $\mid \mid x \mid - \mid y \mid \mid \leq \mid x + y \mid \leq \mid x \ + \mid y \mid$
	\item $\mid x + y \mid = \mid x \mid + \mid y \mid$ when $x$ and $y$ have the same sign, i.e. $x.y \geq 0$
	\item Equations in absolute value must be equal to each other or be negative to each other. For example, if $\mid 3x +1\mid = \mid x-2 \mid $. This implies that $3x + 1 = x -2$ or $3x+1 = -(x-2)$ and the solutions are $x={-3/2,1/4}$
\end{itemize}

\begin{mdframed}[backgroundcolor=yellow]	
	\textbf{Example: Solve $\mid 3x-2 \mid \geq \mid x + 4 \mid$} \\
	\textbf{Solution}
	Lets first identify the critical points (where the value equates to zero) for each side: in this case its $2/3, -4$
	\begin{itemize}
		\item $\mid 3x-2 \mid = {3x-2} \forall x > 2/3$ and $\mid 3x-2 \mid = -(3x-2) \forall x < 2/3$
		\item $\mid x + 4 \mid = (x+4) \forall x > {-4}$ and $\mid x + 4 \mid = -(x+4) \forall x < -4)$
	\end{itemize}
	
	
	For $x < 2/3$, our equation becomes $x + 4 = -(3x-2)$, therefore $x=-1/2$ \\
	For $x > -4$, our equation becomes $x+4 = 3x-2$, therefore $x=3$ \\
	
	Solution is the values for which graph of $\mid 3x-2 \mid$ is above the graph of $\mid x+4 \mid$ \\
	
	\includegraphics[scale=0.5]{solution1}
	
	
\end{mdframed}

	\section{Intervals}
	\textbf{Open Interval}:  If $a$ and $b$ are real numbers and $a<b$,  then the open interval from $a$ to $b$ is written $(a,b)$ , is the collection of all real numbers greater than $a$ and less than $b$, i.e. $(a,b) = {{x | a<x<b}}$ . Geometrically, its represented on the real number line as shown below - the open circles represent the fact that $a$ and $b$ are \textbf{not included} in the interval. \\
	\\
	\includegraphics[scale=0.5]{openinterval}
	\\
	
	\textbf{Closed Interval}:  If $a$ and $b$ are real numbers and $a<b$,  then the closed interval from $a$ to $b$ is written $[a,b]$ , is the collection of all real numbers greater than equal to $a$ and less than equal to $b$, i.e. $[a,b] = {{x | a \leq x \leq b}}$ . Geometrically, its represented on the real number line as shown below - the closed circles represent the fact that $a$ and $b$ are also \textbf{included} in the interval. 	\\
	\\
	\includegraphics[scale=0.5]{closedinterval}
	
	\section{Euclid's Division Lemma}
	\begin{mdframed}[backgroundcolor=yellow]
		Given positive integers $a$ and $b$, there exist unique integers $q$ and $r$ satisfying $a = bq + r$ where $0 \leq r < b$
	\end{mdframed}
	
	\subsection{Highest Common Factor}
	HCF of two positive integers $a$ and $b$ is the largest positive integer $d$ that divides both $a$ and $b$. We can use Euclid’s division lemma to find the HCF of two positive integers. For example, to find the HCF of 455 and 42: \\
	1. Start with highest number, $455 = 42 \times 10 + 35$ \\
	2. Now consider the divisor and remainder: $42 = (35 \times 1) + 7$ \\
	3. Repeat the above step with divisor and remainder, $35 = (7 \times 5) + 0$\
	4. Since the remainder is zero, the divisor at this stage, i..e $7$ will be the HCF \\
	
	\begin{mdframed}[backgroundcolor=yellow]
		$HCF(a,b) \times LCM (a,b) = a \times b$
	\end{mdframed}

	\section{Fundamental Theorem of Arithmetic}
	\begin{mdframed}[backgroundcolor=yellow]
		Every composite (i.e. non-prime) number can be factored as a product of primes, and this factorization is unique, apart from the order in which the prime factors occur.
	\end{mdframed}
	Example:  $32760 = 2^3 \times 3^2 \times 5 \times 7 \times 13$ \\
	\begin{itemize}
		\item $HFC(a,b)$  is the product of smallest power of each “common” prime factor. Example, $6 = 2^1 \times 3^1$ and $20 = 2^2 \times 5^1$ , so the HCF = $2^1$ which is the only common prime factor.
		\item $LCM(a,b)$ is the product of “greatest” power of “each” prime factor involved in the numbers. So in the above example, $LCM(6,20)$ will be $2^2 \times 3^1 \times 5^1 = 60$
	\end{itemize}

	\chapter{Polynomials}	
	
	\section{Introduction}
	Consider $-x^3 + 4x^2 +7x -2$
	\begin{itemize}
		\item \textbf{Terms}: This polynomial has four terms, $-x^3, 4x^3, 7x,-2$
		\item \textbf{Coefficient}: Coefficients of the four terms are $-1,4,7,-2$
		\item \textbf{Exponents should be whole numbers.} For exampple, $1 + 1/x$ is not a polynomial, as the coefficient of the second term is $-1$ which is nota whole number. Another example is $\sqrt{x} + 3$.
		\item \textbf{Constant Polynomial}: Example, $2$ is also a polynomial
		\item \textbf{Zero Polynomial}: Constant $0$ is called the zero polynomial
		\item \textbf{Monomial}: Polynomial with only one term
		\item \textbf{Binomial}: Polynomial with two terms
		\item \textbf{Degree of a polynomial}: Highest power of the variable in a polynomial. For example the degree of $-x^3 + 4x^2 + 7x + 2$ is $3$
		\item Degree of a non-zero constant polynomial is zero, degree of a zero polynomial is not defined
		\item \textbf{Linear Polynomial}: Degree is $1$. For example, $p(x) = 4x + 5$ is a linear polynomial and is of the form $ax + b$, can contain a maximum of two terms.
		\item \textbf{Quadratic Polynomial}: Degree is 2, is of the form $ax^2 + bx + c$ and will contain a maximum of three terms. In general, a polynomial of degree $n$ will contain a maximum of $n+1$ terms.
		\item \textbf{Zero of a Polynomial}: Consider $p(x) = x -1$. When we equate $p(x)$ to $0$, we get $x=1$, and $1$is the root (or zero) of the polynomial. 
		\item Non-zero constant polynomial has no zero
		\item Every real number is a zero of the zero polynomial
		\item A linear polynomial has one and only one zero \\
	\end{itemize}
In general, a polynomial on one variable $x$ of	degree $n$ is an expression of the form: \\
	\begin{mdframed}[backgroundcolor=yellow]
		$p(x) = a_nx^n + a_{n-1}x^{n-1} + ............ + a_1x + a_0$
	\end{mdframed}
where $a_0, a_1,...$ are constants and $a_n \neq 0$.
		\subsection{Graphs of Polynomial Functions}
		\begin{itemize}
			\item The graph of a polynomial function is continuous. Essentially, this means that the graph of a polynomial function has no breaks, holes, or gaps. Informally, you can say that a function is continuous when its graph can be drawn with a pencil without lifting the pencil from the paper.
			\item Another feature of the graph of a polynomial function is that it has only smooth, rounded turns. It cannot have a sharp turn such as the one shown below.
		\end{itemize}
	
		\includegraphics[scale=0.5]{polygraph1} \\
		\includegraphics[scale=0.5]{polygraph2} \\
		
		
	\section{Remainder Theorem}
	If $p(x) $and $g(x)$ are two polynomials such that degree of $p(x)$ $>$ degree of $g(x)$ and $g(x) \neq 0$, then we can find polynomials $q(x)$ and $r(x)$ such that:
	
	$$p(x) = g(x)q(x) + r(x)$$ \\
	
where $r(x) =0$ or degree of $r(x)$ $<$ degree of $g(x)$. Essentially, what this means is $p(x)$ divided by $g(x)$ gives us $q(x)$ as the quotient and $r(x)$ as the remainder.
	\begin{mdframed}[backgroundcolor=yellow]
		\textbf{Remainder Theorem}:  Let $p(x)$ be any polynomial of degree $\ge 1$ and let $a$ be any real number. If $p(x)$ is divided by $x-a$, then the remainder is $p(a)$
	\end{mdframed}	

	\section{Factor Theorem}
	\begin{mdframed}[backgroundcolor=yellow]
		\textbf{Factor Theorem}:If p(x) is a polynomial of degree $n \ge 1$ and $a$ is any real number, then:
		\begin{itemize}
			\item $x-a$ is a factor of $p(x)$ if $p(a)=0$, and
			\item $p(a)=0$ if $x-a$ is a factor of $p(x)$
		\end{itemize} 
	\end{mdframed}
	
	\section{Useful Identities}
	\begin{itemize}
		\item $(x+y)^3 = x^3 + y^3 + 3xy(x+y)$
		\item $(x-y)^3 = x^3 - y^3 -3xy(x-y)$
		\item $x^3 + y^3 + z^3 -3xyz = (x+y+z)(x^2 + y^2 + z^2 - xy -yz -zx)$
		\item $(x+y+z)^2 = x^2 + y^2 + z^2 + 2(xy + yz + zx)$
		\item $(x^3 + y^3)= (x+y)(x^2 -xy + y^2)$
		\item $x^3 - y^3 = (x-y)(x^2 + xy + y^2)$
	\end{itemize}
	\section{Synthetic Division}
	Synthetic division is used to divide one polynomial, $P(x)$ by a polynomial of the form $x-k$, where the coefficient of $x$ is $1$. It involves the below steps:
	\begin{enumerate}
		\item Arrange $P(x)$ in descending powers of $x$. If a term is missing, write a $0$ for its coefficient.
		\item Place $k$, the additive inverse of $-k$ in the divisor.
		\item Bring down the leading coefficient of largest power of $x$ to the 3rd row.
		\item Multiply the leading coefficient we bought down by $k$, place the product in the next position under the 2nd coefficient of the dividend and add the two numbers
		\item Multiply this sum by $k$, place the product under the 3rd coefficient and add.
		\item Continue this process until all coefficients in the dividend are used.
	\end{enumerate}

\begin{mdframed}[backgroundcolor=yellow]
	\textbf{Example: Divide $P(x) = 2x^4 - 3x^3 -4x + 8$ by $x-3$}
\end{mdframed}
\textbf{Solution: } : Using Synthetic Division, we can see from the below figure that the quotient is $2x^3 + 3x^2 + 9x + 23$ and remainder is $77$ \\

\includegraphics[scale=0.7, width=\linewidth, frame]{sdivision}

	\section{Zeros of a Polynomial}
	
	\begin{mdframed}[backgroundcolor=yellow]
		The zeros of a polynomial are the solutions to the equation $p(x) = 0$, where p(x) represents the polynomial. If we graph this polynomial as $y = p(x)$, then you can see that these are the values of $x$ where $y = 0$. In other words, they are the x-intercepts of the graph.
		\textbf{In general, given a polynomial $p(x)$ of degree $n$, will have a maximum of $n$ zeroes}
	\end{mdframed}

	\chapter{Graphs of Polynomial Functions}
	\section{Linear Polynomial - Straight Line}
	A linear polynomial is of the form $p(x) = mx + c$, and its graph is a straight line. Here $m$ is the slope or "steepness" of this straight line, i.e. how quickly it rises (or falls) as we move along the $x-axis$ and the y-intercept is $(0,c)$. The slope of a line is the ratio of $rise/run$:
	
	\includegraphics[scale=0.7]{slope1}

	The slope m of a nonvertical line that passes through the points $A(x_1,y_1)$ and $B(x_2,y_2)$ is
	
	$$m = rise/run = (y_2-y_1)/(x_2-x_1)$$
	
	\begin{itemize}
		\item A horizontal line has slope zero.
		\item The slope of a vertical line is not defined
		\item If two lines are parallel, they have the same slope
		\item If two lines are perpendicular to each other, the product of their slopes is $-1$
		\item Graph of $y = ax + b$ is a straight line which intersects the x-axis at exactly one point, $(-b/a,0)$. This means that a linear polynomial has exactly one zero, namely the point where the graph intersects the x-axis.
		\item Figure below shows several lines with their slopes - lines with positive slope slant upward to the right, whereas lines with negative slope slant downward to the right.
	\end{itemize}
	
	\includegraphics[scale=0.7]{slope2}
	
	\section{Quadratic Polynomial - Parabola}

	A quadratic polynomial can be written in the form $p(x) = ax^2 + bx + c$ where $a \neq 0 $ and $a,b,c$ are constants. \\
	The \textbf{standard form} of a quadratic function is $p(x) =a(x-h)^2 + k, a \neq 0$ \\
	The graph of a quadratic function is called a parabola
	\subsection{Leading Coefficient}
	\begin{itemize}
		\item In either form: If leading coefficient, $a>0$ then the parabola opens up. If leading coefficient, $a<0$ then the parabola opens down.
		\item The leading coefficient indicates how "fat" or how "skinny" the parabola will be:
		\begin{itemize}
			\item If  $\mid a \mid > 1$ the parabola will be "skinny" because it grows more quickly
			\item If $\mid a \mid < 1$,  the parabola will be "fat" because it grows more slowly.
		\end{itemize}
	\end{itemize}
	
	\includegraphics[scale=0.6]{parabolalc}
	
	\subsection{Vertex}
	The vertex is the lowest or highest point (depending on direction) on the graph of a quadratic function.
	\begin{itemize}
		\item In the non-standard form, i.e. $p(x) = ax^2 + bx + c$, the vertex is at the coordinate $-b/2a, p(-b/2a)$.
		\item In the standard form, $p(x) = a(x-h)^2 + k, a \neq 0$ the vertex is at the coordinate $(h,k)$.
		\item Let's assume we need to find the vertex of $p(x) = -(x-5)(x-9)$ which is in the factored form:
		\begin{enumerate}
			\item Determine the zeros, in this case it is $-2,4$
			\item Determine the x-coordinate of the vertex by averaging the zeros, $(-2+4)/2 = 1$
			\item Determine the y-coordinate of the vertex by substituting the x-coordinate of vertex and solving for y, $y = (1+2)(1-4)=-9$
			\item Vertex is (1,-9)
		\end{enumerate} 
		
		
	\end{itemize}
	
	
	\subsection{Axis of Symmetry}
	Each parabola is symmetric about a vertical line called the axis of symmetry.  This vertical line goes through the vertex.
	\begin{itemize}
		\item In the non-standard form, i.e. $p(x) = ax^2 + bx + c$, the line of symmetry is represented by the vertical line $x= -b/2a$
		\item In the standard form, $p(x) = a(x-h)^2 + k, a \neq 0$ the line of symmetry is the vertical line $x = k$ 
		\item In the factored form, $p(x) = a(x-r_1)(x-r_2)$, the line of symmetry is the vertical line $x = (r_1 + r_2)/2$ 
	\end{itemize}
	
	\includegraphics[scale=0.6]{parabolalos}
	
	\subsection{Discriminant}
	In the polynomial $p(x) = ax^2 + bx + c$, the $D = b^2 - 4ac$ is the discriminant.
	\begin{table}[ht]
		\begin{tabular}{|l| l| l|  }
			\hline
			\textbf{a} & \textbf{D} & \textbf{p(x)}\\
			\hline
			$a>0$ & $D<0$ & $p(x) >0 \forall x$\\
			\hline
			$a<0$ & $D<0$ & $p(x) >0 \forall x$\\
			\hline
			$a>0$ & $D=0$ & $p(x) >0 \forall x$ except at vertex.\\
			\hline
			$a>0$ & $D>0$ & $p(x)$ has two real roots $\alpha, \beta$, where $\alpha < \beta$. $p(x) > 0 \forall x \in (-\infty, \alpha) \cup (\beta, \infty)$. $p(x) < 0 \forall x \in (\alpha, \beta)$ \\
			\hline
			$a<0$ & $D=0$ & $p(x) < 0 \forall x$ except at the vertex \\
			\hline
			$a<0$ & $D>0$ &  $p(x)$ has two real roots $\alpha, \beta$, where $\alpha < \beta$. $p(x) < 0 \forall x \in (-\infty, \alpha) \cup (\beta, \infty)$. $p(x) > 0 \forall x \in (\alpha, \beta)$ \\
			\hline
		\end{tabular}
	\end{table}
	
	Please refer to the figure in the next page for the graphs associated with these conditions.
	
	\includegraphics[scale=0.7]{parabolad} 
	
	\section{Graphs of Other Polynomial Functions}
	
	We will consider the polynomial function $f(x) = a_nx^n + a_{n-1}x^{n-1} + .....+ a_1x + a_0$, where $a_n \neq 0$ and the leading term is $a_nx^n$
	
	\subsection{Symmetry}
	
	\begin{itemize}
		\item If the function is an even function, that is, $f(–x) = f(x)$, its graph is symmetric with respect to the y-axis.
		\item If the function is an odd function, that is, $f(–x) = –f(x)$, its graph is symmetric with respect to the origin.
	\end{itemize}
	
	\includegraphics[scale=0.5]{symmetry}
	
	\subsection{End Behavior - Leading Coefficient and Degree}
	
	The graph of a polynomial function will either ultimately rise or fall as $x$ increases without bound and will either rise or fall as $x$ decreases without bound. This is because for very large or very small inputs, the \textbf{ leading coefficient dominates the size of the output}. We call this behavior the \textbf{end behavior} of a function.
	
	\includegraphics[scale=0.5]{evenodd1}
	
	\includegraphics[scale=0.5]{evenodd2}
	
	
	\begin{table}[ht]
		\begin{tabular}{|l| l| l|  }
			\hline
			\textbf{$n$} & \textbf{$a_n$} & Graph\\
			\hline
			Odd & Positive & Graph falls to the left and rises to the right\\
			\hline
			Odd & Negative & Graph rises to the left and falls to the right\\
			\hline
			Even & Positive & Graph rises to the left and to the right \\
			\hline
			Even& Negative & Graph falls to the left and to the right \\
			\hline
		\end{tabular}
	\end{table}

	Refer to the below figure for examples:-

	\includegraphics[scale=0.7]{polynomialg}
	
	\subsection{Zeroes and their multiplicities}
	If a polynomial contains a factor of the form $(x-h)^p$, the behavior near the x-intercept $h$ is determined by the power $p$. We say that $x=h$ is a zero of multiplicity $p$.
	
	For example, consider the polynomial $p(x) = 3(x+5)^3(x+2)^4(x-1)^2)(x-5)$. The multiplicity of each zero is the number of times that its corresponding factor appears:
	
	\begin{itemize}
		\item $x=-5$ has an odd multiplicty of 3
		\item $x=-2$ has an even multiplicity of 4
		\item $x=1$ has an even multiplicity of 2
		\item $x=5$ has an odd multiplicity of 1
	\end{itemize}
	
	\textbf{\hl{The graph of a polynomial function will touch the x-axis at zeros with even multiplicities. The graph will cross the x-axis at zeros with odd multiplicities.}} \\
	
	The point of multiplicities with respect to graphing is that any factors that occur an even number of times are squares, so they don't change sign. Squares are always positive. This means that the x-intercept corresponding to an even-multiplicity zero can't cross the x-axis, because the zero can't cause the graph to change sign from positive (above the x-axis) to negative (below the x-axis), or vice versa.  An even-multiplicity zero makes the graph just barely touch the x-axis, and then turns it back around the way it came. \hl{This means that the sign of $p(x)$ does not change on either side of zero with even multiplicity.}
	
	\begin{itemize}
		\item If the graph crosses the x-axis and appears almost linear at the intercept, it is a single zero
		\item If the graph touches the x-axis and bounces off of the axis, it is a zero with even multiplicity. For higher even powers, such as 4, 6, and 8, the graph will still touch and bounce off of the x-axis, but for each increasing even power the graph will appear flatter as it approaches and leaves the x-axis
		\item If the graph crosses the x-axis at a zero, it is a zero with odd multiplicity. For higher odd powers, such as 5, 7, and 9, the graph will still cross through the x-axis, but for each increasing odd power, the graph will appear flatter as it approaches and leaves the x-axis.
		\item The sum of the multiplicities is the degree of the polynomial function
	\end{itemize}
	
	\includegraphics[scale=0.5]{zeromul}
	
	Polynomial zeroes with even and odd multiplicities will always behave in this way. From the graphs below, we can see that for higher even or odd powers, the graph still either bounces or crosses the x-axis, but it becomes flatter as it approaches and leaves the x-axis. \\
	
	\includegraphics[scale=0.6]{oddeven1}
	
	Suppose for example we graph the function $f(x) = (x+3)(x-2)^2(x+1)^3$
	
	\begin{itemize}
		\item At $x = -3$, we have a single zero (i.e. multiplicity is 1) and the factor is linear, so the behavior near the intercept is like that of a line; it passes directly through the intercept
		\item At $x = 2$, multiplicity is 2 - graph touches the axis at the intercept and changes direction. The factor is quadratic (degree 2), so the behavior near the intercept is like that of a quadratic—it bounces off of the horizontal axis at the intercept.
		\item At $x = -1$, multiplicity is 3. The graph passes through the axis at the intercept but flattens out a bit first. This factor is cubic (degree 3), so the behavior near the intercept is like that of a cubic with the same S-shape near the intercept
	\end{itemize}
	
	\includegraphics[scale=0.5]{graphex}
	
	\subsection{Turning Points}
	From the previous subsection we can see that the graph of a polynomial consists of a smooth line with a series of hills and valleys called turning points. 
	\begin{itemize}
		\item The maximum number of turning points is one less than the degree of the polynomial
		\item The point where the graph has a turning point, the derivative of the function/polynomial (i.e tangent/slope at that point will be parallel to x-axis) becomes zero, which provides the point of local minima or maxima.
	\end{itemize}
	
	Consider the below graph for example
	
	\begin{itemize}
		
		\item Tangent to the curve at point for which $x < -1$ and $x > 1$ makes an acute angle with the positive direction of x-axis, hence derivative is positive for these points.
		\item For $-1 < x < 1$, tangent to the curve makes an obtuse angle with the positive direction of x-axis, hence derivative is negative at these points.
		\item At $x=1$ and $x=-1$, the tangent is parallel to x-axis, where derivative is zero.
	\end{itemize}
	
	\includegraphics[scale=0.6]{turningpoint}
	
	\subsection{Intermediate Value Theorem}
	\hl{The sign of a polynomial between any two \textbf{consecutive zeros} is either always positive or always negative}. This is because polynomial functions are continuous functions (no breaks in the graph), which means that the only way to change signs is to cross the x-axis. But if this happened, the given zeros would not be consecutive! It is not necessary, however, for a polynomial function to change signs between zeros.
	\\
	
	\includegraphics[scale=0.5]{pinterval1}
	
	Let's consider the polynomial $p(x) = (x+3)(x-1)^2$. The zeros are $-3,1$. This creates three intervals over which the sign of $p(x)$ is constant:
	\\
	
	\includegraphics[scale=0.5]{pinterval2} \\
	
	\includegraphics[scale=0.40]{pinterval3} \\
	
	\begin{mdframed}[backgroundcolor=yellow]
		\textbf{Intermediate Value Theorem} If $y = f(x)$ is a continuous curve on the interval $[a,b]$, and $w$ is a number between $f(a)$ and $f(b)$, then there must be at least one value, $c$, within $[a,b]$ such that $f(c)=w$. \\
		
		\includegraphics[scale=0.6]{ivt}
	\end{mdframed} 
	
	We can use the special case of Intermediate Value Theorem for graphing polynomials - \textbf{If there are two points are on opposite sides of the x-axis, we can confirm that there is a zero between them}. If a point on the graph of a continuous function $f(x)$ at $x=a$ lies above the x-axis and another point at $x=b$ lies below the x-axis, there must exist a third point between $x=a$ and $x=b$ where the graph crosses the x-axis. \textbf{In other words, the Intermediate Value Theorem tells us that when a polynomial function changes from a negative value to a positive value, the function must cross the x-axis}. The figure below shows that there is a zero between $a$ and $b$. In general, if f$(a)$ and $f(b)$ have opposite signs, then there exists at least one value $c$ between $a$ and $b$ for which $f(c)=0$. \\
	
	\includegraphics[scale=0.6]{ivt1}
	
	
	\subsection{Concavity - Curve Sketching using Differentiation}
	
	A local \textbf{maximum} occurs when the first derivative = 0 and it changes sign from positive to negative (as we go left to right):
	
	\includegraphics[scale=0.5]{localmaximum}
	
	A local \textbf{minimum} occurs when first derivative = 0 and it changes sign from negative to positive
	
	\includegraphics[scale=0.5]{localminimum}
	
	\begin{itemize}
		\item We say that a function $y = f(x)$ is concave up (CU) on a given interval if the graph of the function always lies above its tangent lines on that interval. In other words, if you draw a tangent line at any given point, then the graph seems to curve upwards, away from the line.
		\item Conversely, a function is concave down (CD) on a given interval if the graph of the function always lies below its tangent lines on that interval. That is the graph seems to curve downwards, away from its tangent line at any given point.
	\end{itemize}
	
	\includegraphics[scale=0.6]{concave1} \\
	
	An \textbf{Inflection Point} is a point in a graph at which the concavity changes, i.e. where a curve changes from Concave upward to Concave downward (or vice versa): \\
	
	\includegraphics[scale=0.6]{inflection1} \\
	
	Inflection points may be difficult to spot on the graph itself, so we must rely on calculus to find them:
	
	\begin{itemize}
		\item When $f^{''}(x) > 0 $ on an interval, the function is \textbf{concave upward} on that interval.
		\item When $f^{''}(x) < 0 $ on an interval,  the function is \textbf{concave downward} on that interval
		\item When $f^{''}(x)$ changes sign (from positive to negative, or from negative to positive) at a point $x = c$, then there is an inflection point located at $x = c$ on the graph. In particular, the point $(c, f(c))$ is an inflection point for the function $f$. 
	\end{itemize}
	
	\begin{mdframed}[backgroundcolor=yellow]
		Not all solutions of $f^{''}(x)=0$ are going to be inflection points. We have to find out how the concavity changes from interval to interval. Once you determine the roots of  $f^{''}(x)=0$, find out the sign (negative or positive) of  $f^{''}(x)$ in those intervals.\\
	\end{mdframed} 
	
	
	Consider $f(x) = x^6/30 - x^5/20 - x^4 + 3x + 20$ \\
	We see that $f^{''}(x) = x^4 - x^3-12x^2 = x^2(x-4)(x+3)$
	There are three solutions, $x = 0, 4, and -3$. These are not necessarily all going to be inflection points, though! We have to find out how the concavity changes from interval to interval first. \\
	
	\includegraphics[scale=0.5]{concavex1} \\
	
	Notice that there is no change in concavity at x = 0, and the only inflection points are at x = -3 and 4: \\
	
	\includegraphics[scale=0.5]{concaveex2}
	
	
	\chapter{Quadratic Equations}
	
	\section{Introduction}
	When a quadratic polynomial, $p(x) = ax^2 + bx + c$ is equated to zero, we get a quadratic equation. Therefore, a quadratic equation is of the form $ax^2 + bx + c = 0$ where $a,b,c$ are Real Numbers and $a \neq 0$. \\
	$b^2 -4ac$ is called the Discriminant(D):
	\begin{itemize}
		\item If $D \ge 0$ then the roots are real, ${(-b \pm \sqrt{D})}/2a$
		\item If $D=0$, then we have two equal roots, $-b/2a$
		\item If $D<0$, then there are no real roots and the graph \textbf{does not} cut the x-axis.
		\item If $D>0$, then there are two \textbf{distinct} real roots and the graph will cut the x-axis at two places.
	\end{itemize}
	
	When coefficients $a,b,c$ are rational or integers:
	\begin{itemize}
		\item $D>0$ and its a perfect square: Roots are rational.
		\item $D>0$ and an imperfect square: Roots are irrational and exist as \textbf{conjugate pairs}.
	\end{itemize}
	
	\begin{mdframed}[backgroundcolor=yellow]
		Note 1: If $\alpha, \beta$ are the roots of a quadratic equation, then the equation can be written in the following ways:
		\begin{itemize}
			\item $(x - \alpha)(x - \beta) = 0$ (by factor theorem)
			\item $x^2 - (\alpha + \beta)x + \alpha\beta$ : This is useful when we know the sum and product of roots.
		\end{itemize}
	\end{mdframed}
	\begin{mdframed}[backgroundcolor=yellow]
		Note 2: If $\alpha, \beta$ are the roots of $ax^2 + bx + c = 0$, then:
		\begin{itemize}
			\item Sum of Roots, $\alpha + \beta = {(-b/2a)}$
			\item Product of Roots: $\alpha\beta = c/a$
		\end{itemize}
	\end{mdframed}
	
	\section{Common Roots}
	Let's consider the following quadratic equations:
	$$a_1x^2 + b_1x + c_1 = 0$$
	$$a_2x^2 + b_2x + c_2 = 0$$
	
	\subsection{Condition for one common root}
	If $\alpha$ is the common root for the above equations, then:
	$$a_1\alpha^2 + b_1\alpha + c_1 = 0$$
	$$a_2\alpha^2 + b_2\alpha + c_2 = 0$$
	
	Solving these two equations by cross-multiplication, we get the condition for one common root
	
	\begin{mdframed}[backgroundcolor=yellow]
		$$\alpha^2 / {(b_1c_2 - b_2c_1)} = \alpha/{(c_1a_2-c_2a_1) = 1/{(a_1b_2 - a_2b_1)}}$$
	\end{mdframed}
	
	\subsection{Condition for both roots to be common}
	Let $\alpha, \beta$ be the roots common to the above equation, which means that both equations are identical - the condition for this is:
	
	\begin{mdframed}[backgroundcolor=yellow]
		$$a_1/a_2 = b_1/b_2 = c_1/c_2$$
	\end{mdframed}
	
	\section{Location of Roots}
	For a given quadratic equation $ax^2 + bx + c = 0$, where $a>0$, the Discriminant(D) = $b^2-4ac$, we can have the following conditions:
	\subsection{Both roots are greater than a number $d$}
	All the conditions below must be true  for both roots $\alpha, \beta$ to be greater than a number $d$:
	\begin{itemize}
		\item $D>0$
		\item Line of Symmetry, $-b/2a > d$
		\item Value of $f(x) = ax^2 + bx + c$ at $d$ will be greater than $0$ 
	\end{itemize}
	
	\includegraphics[scale=0.6]{condition1}
	
	\subsection{Both roots lie on opposite sides of a number $d$}
	In this case we cannot comment on the line of symmetry, which can lie on either side of $d$. Below conditions must be true:
	\begin{itemize}
		\item $D>0$
		\item Value of $f(x) = ax^2 +bx + c$ at $d$ is less than $0$.
	\end{itemize}
	
	\includegraphics[scale=0.6]{condition2}
	
	\subsection{Both roots lie between $(d,e)$}
	\begin{itemize}
		\item $D>0$
		\item Line of Symmetry, $d < -b/2a < e$
		\item Value of $f(x) = ax^2 + bx + c$ at $d$ and $e$ is greater than zero.
		
		\includegraphics[scale=0.6]{condition3}
	\end{itemize}
	
	\subsection{$(d,e)$ lie between the roots}
	\begin{itemize}
		\item Value of $f(x) = ax^2 + bx + c$ at
		/\ $d$ and $e$ is less than zero.
	\end{itemize}
	
	\includegraphics[scale=0.6]{condition4}
	
	\subsection{Exactly one root lies in interval $(d,e)$}
	\begin{itemize}
		\item $f(d)f(e) <0$, where $f(d),f(e)$ are values of $f(x) = ax^2+bx+c$ at $d,e$
		\item $f(d)f(e)=0$ - In this case, verify for extraneous roots that are not applicable
	\end{itemize}
	
	\includegraphics[scale=0.6]{condition5}
	
	\chapter{Logarithms}
	\section{Introduction}
	Exponent Form: $6^2 = 36$ \\
	Logarithmic Form: $log_6{36} = 2$
	\begin{mdframed}[backgroundcolor=yellow]
		In general, $log_aN = x$, where:
		\begin{itemize}
			\item Base $a>0$
			\item Base, $a \neq 1$
			\item $N > 0$ and $N \in R^+$, i.e. $N$ is a positive real number
		\end{itemize}
	\end{mdframed}
	\section{Identities}
	
	\begin{itemize}
		\item $log_NN = 1$
		\item In $log_aN$, if $a \times N =1$, (i.e. if $a$ and $N$ are reciprocal to each other) then it means that $logaN = -1$ 
		\item $log_a1 = 0$
		\item $log_aN + log_aM = log_a{NM}$
		\item $log_aN - log_aM = log_a{(N/M)}$
		\item $log_aN^\beta = \beta log_aN$
		\item$x^{log_xN} = N$
		\item $\log_{a^q}n^p = (p/q)log_an, a,n > 0, a \ne 1$
		\item $a^{log_bc} = c^{log_ba}$
		\item \textbf{Change of Base Formula}: Assuming $x,a,b >0$ and $ a \ne1, b \ne 1$, $log_ax = log_bx/log_ba$. Example, $log_{16}32 = log_2(32)/log_2(16) = 5/4$
		\item $\ln x = log_ex$ , Natural Logarithm (Base $e$). The number $e$ is defined as the value that $(1 + 1/n)^n$ approaches as $n$ becomes large. The approximate value of $e$ is $2.718$,and it's an irrational number.
		\item \textbf{Base Switch Rule}: $log_bc = 1/log_cb$. Example, $1/log_82 =log_28$
		\item $\ln x = y \leftrightarrow e^x = y$
		\item $log_{10}x$ , Common Log (Base $10$)
	\end{itemize}
	
	Here is the graph of $log_2x$: 
	\includegraphics[scale=0.6]{loggraph}
	
	\section{Logarithmic Inequalities}
	\begin{itemize}
		\item If $log_ax > log_ay$, this implies that $x>y \: if \: a>1$ or $0<x<y \: if \: 0<a<1$
		\item If $log_ax >y$, this implies that $x>a^y \: if \: a>1$ or $0<x<a^y \: if \: 0<a<1$
		\item $log_ax > 0$ implies $x>1$ and $a>1$ or $0<x<1$ and $0<a<1$
	\end{itemize}

	\chapter{Exponential Functions}
	\section{Introduction}
	\begin{mdframed}[backgroundcolor=yellow]
		An exponential function with base $a$ is defined for all real numbers $x$ by $f(x) = a^x$ where $a>0$ and $a \neq 1$
	\end{mdframed}
	
	\section{Graphs of Exponential Functions}
	 Figure below shows the graphs of the family of exponential functions $f(x) = a^x$ for various values of the base $a$. All of these graphs pass through the point $(0,1)$ because $a^0 = 1$ for $a \neq 1$.
	 
	 \begin{itemize}
	 	\item If $0<a<1$, the exponential function \textbf{decreases} rapidly.
	 	\item If $a>1$, the exponential function \textbf{increases} rapidly.
	 \end{itemize}
	 
	 \includegraphics[scale=1.0]{expgraph}
	
	\section{Transformation of Exponential Functions}
	 \subsection{$f(x)=1+a^x$}
	 To obtain the graph of $1+a^x$, we start with the graph of $f(x) = a^x$ and shift it upward 1 unit.
	 
	 \includegraphics[scale=0.8]{exp1}
	 
	 \subsection{$f(x) = -a^x$}
	 We start with the graph of $f(x) = a^x$, but here we reflect in the x-axis
	 
	 \includegraphics[scale=0.8]{exp2}
	 
	 \subsection{$f(x) = a^{-x}$}
	 In this case, the graph is the mirror image of $f(x) = a^x$
	 
	 \includegraphics[scale=0.8]{exp3}
	 
	 \section{The Natural Exponential Function}
	 The number $e$ is defined as the value that $(1+1/n)^n$ approaches as $n$ becomes large. The approximate value is $e = 2.71828$. The number $e$ is the base for natural exponential function, which in certain applications is easier to work with than $log_{10}$. \\
	 The \textbf{natural exponential function} is the function $f(x) = e^x$ with base $e$. Since $2 < e < 3$, the graph of $e^x$ lies between the graphs of $2^x$ and $3^x$:\\
	 \\
	 \includegraphics[scale=0.8]{exp4}
	 
	 \chapter{Set Theory, Relation and Functions}
	 \section{Introduction}
	 \begin{itemize}
		\item A set is a well defined collection of objects. "Well defined" means that given an element, we can determine whether it belongs to the set or not.
		\item Sets are denotes by capital letters and its elements in small letters. If $a$ is an element of set $A$, we say that "a belongs to A", i.e. $a \in A$.
		\item There are two methods of representing a set:
		\begin{itemize}
			\item \textbf{Roster or Tabular Form}: In roster form, all the elements of a set are listed, the elements are being separated by commas and are enclosed within braces. For example, the set of all even positive integers less than 7 will be described in roster form as ${\{2,4,6\}}$. The order in which the elements are listed is immaterial. Also, all elements are taken as distinct, for example the set of letters that make up the word "SCHOOL"will be $\{S,C,H,O,L\}$
			\item \textbf{Set-Builder Form}: In this form, all the elements of a set possess a single common property which is not possessed by elements outside the set. For example, the set of vowels in the English alphabet will be written as $V = \{x: x \: is \: a \: vowel \: in\: English \: alphabet\}$. As we can see, in this form, we describe the element of the set by using a symbol (x in this case), followed by a colon. After the colon, we write the propeerty possessed by the elements of the set. In this description the braces stand for "the set of all" and the colon stands for "such that". Here is another example where we are representing a set of square of natural numbers: $A = \{x: x = n^2, \: where \: n \in N \}$
		\end{itemize}
		\item \textbf{Empty Set}: A set that does not contain any element is called empty, null or void set, and is denoted by the symbol $\phi$ or $\{\}$
		\item \textbf{Non-Empty Finite Set}: By number of elements of a set $S$, we mean the number of distinct elements of the set and we denote it by $n(S)$. If $n(S)$ is a natural number, then $S$ is a non-empty finite set. A set which is empty of consists of a definite number of elements is called \textbf{finite} otherwise, the set it called \textbf{infinite}.
		\item \textbf{Equal Sets}: Two sets $A$ and $B$ are said to be equal if they have exactly the same elements, and we write $A=B$. Note: A set does not change if  one or more elements of the set are repeated. For example, the sets $A=\{1,2,3\}$ and $B=\{2,2,1,3,3\}$ are equal. That's why we do not repeat elements in describing a set.
		\item \textbf{Subsets}: Set $A$ is a subset of set $B$ if every element of $A$ is also an element of $B$. In other words, $A \subset B$ if whenever $a \in A$, then $a \in B$. If $A$ is not a subset of $B$, we write $A \not\subset B$. It follows that every $A$ is a subset of itself, i.e $A \subset A$. Since $\phi$ has no elements, we say that it is a subset of of every set.
		\item \textbf{Power Set}: Consider the set $\{1,2\}$ - below are all of its subsets:
		\begin{enumerate}
			\item $\phi$ is a subset of all every set, so $\phi$ is a subset
			\item $\{1\}$
			\item $\{2\}$
			\item We also know that every set is its own subset, so $\{1,2\}$ is also a subset.
		\end{enumerate}
		The set of all these subsets is called the power set of $\{1,2\}$. The collection of all subsets of set $A$ is called the power set of $A$ and is denoted by $P(A)$. In $P(A)$, every element is a set. Also, \hl{if $A$ is a set with $n(A) =m$, then $n[P(A)]=2^m$}.
		\item \textbf{Universal Set}: A Universal Set is the set of all elements under consideration, denoted by $U$- all other sets are subsets of the universal set. Think of a Universal set as the "big picture" - it includes everything under consideration, or everything that is relevant to the problem you are trying to solve. In Venn diagrams, the universal set is usually represented by a rectangle. For example, in human population studies, the universal set consists of all the people in the world.
		\end{itemize}
		\section{Venn Diagrams}
		Venn diagrams are used to represent relationships between sets - the universal set is represented by a rectangle and its subsets by circles. The elements of the sets are written in their respective circles. For example, if $U = \{1,2,3,....10\}$ is the universal set of which $A=\{2,4,6,8,10\}$ is a subset, the Venn diagram is shown below: \\
		\includegraphics[scale=0.8]{set1}
		
		\section{Operations on Sets}
		\subsection{Union of Sets}
		\begin{mdframed}[backgroundcolor=yellow]
			Let $A$  and $B$ be any two sets. The union of  $A$  and $B$ is the set which consists of all the elements of $A$  and all the elements of $B$, the common elements being taken only once, and is denoted by $\cup$, read as "A union B". \\
			In symbols, we write $A \cup B = \{x : x\in A \: or \: x \in B\}$. Below figure shows the Venn diagram representation - the shaded portion in Blue reprsents $A \cup B$.\\
			
			\includegraphics[scale=0.8]{set2}\\
		\end{mdframed}
		Example, let $A = \{2,4,6,8\}$ and $B = \{6,8,10,12\}$, then $A \cup B = \{2,4,6,8,10,12\} $ 
		
		\subsection{Intersection of Sets}
		\begin{mdframed}[backgroundcolor=yellow]
			The intersection of sets $A$ and $B$ is the set of all elements which are \textbf{common} to both $A$ and $B$, and is represented by $A \cap B$ . The intersection of two sets A  and B is the set of all those elements which belong to both A  and B, and is written as $A \cap B = \{ x: x \in A \: and \: x \in B\}$. The shaded portion in the below figure indicates the intersection of $A$  and $B$.\\
			\includegraphics[scale=0.8]{set3}
		\end{mdframed}
	
		Example: Let $A = \{1,2,3,4,5,6,7,8,9,10\}$ and $B = \{2,3,5,11,12\}$, then $A \cap B = \{2,3,5\}$. \\
		
		If $A$ and $B$ are two sets such that $A \cap B = \phi$, then $A$ and $B$ are called \textbf{Disjoint Sets}, i.e. there are no common elements between them.
		
		\subsection{Difference of Sets}
		
		\begin{mdframed}[backgroundcolor=yellow]
			The difference of set $A$ and $B$ is the set of elements which belong to $A$ but not to $B$. \\
			\includegraphics[scale=0.6]{set4}
		\end{mdframed}
		For example, let $A = \{1,2,3,4,5,6\}$ and $B = \{2,4,6,8\}$. Therefore $A-B = \{1,3,5\}$ and $B-A = \{8\}$
		
		\subsection{Complement of a Set}
		\begin{mdframed}[backgroundcolor=yellow]
		Let $\cup$ be the universal set and $A$ be its subset. The complement of $A$, written as $A^{'}$ is the set of all elements of $\cup$ which are NOT in $A$. \\
		\includegraphics[scale=0.6]{set5}
		\end{mdframed}
		
		\section{Ordered Pair and Cartesian Product of Sets}
		
		In a set, the order of the elements is irrelevant, so $\{a,b\} = \{b,a\}$. If the order of elements is relevant, then we use a different object called \textbf{Ordered Pair.} An ordered pair is a pair of objects where one element is designated first and the other element is designated second, denoted $(a,b)$. Two ordered pairs are equal, if and only if the corresponding first elements are equal and the second elements are also equal.
		
		\begin{mdframed}[backgroundcolor=yellow]
			The cartesian product of two sets $A$ and $B$, denoted $A X B$ is the set of all possible \textbf{ordered pairs} where the elements of $A$ are first and the elements of $B$ are second. \\
			In set-builder notation, $A X B = \{(a,b): a \in A \: and \: b \in B\}$ \\
			Note that:
			\begin{itemize}
				\item A X B $\neq$ B X A
				\item If there are $p$ elements in $A$ and $q$ elements in $B$, then there will be $pq$ elements in $AXB$, i.e. if $n(A)=p, n(B)=q$ then $n(AXB)=pq$
				\item A X $\phi$ = $\phi$,since no ordered pairs can be formed when one of the sets is empty.
			\end{itemize}
		\end{mdframed}
		Example, let $A$ = $\{H,T\}$ and $B = \{1,2,3,4,5,6\}$ \\
		$A X B = \{H, 1), (H, 2), (H, 3), (H, 4), (H, 5), (H, 6), (T, 1), (T, 2), (T, 3), (T, 4), (T, 5), (T, 6)\}$ \\
		$B X A = \{(1, H), (2, H), (3, H), (4, H), (5, H), (6, H), (1, T), (2, T), (3, T), (4, T), (5, T), (6, T)\}$\\
		
		\section{Relations}
		A relation \textbf{R} from a non-empty set $A$ to a non-empty set $B$ is a subset of the cartesian product $A X B$. Let $R \subset A X B$ means that \textbf{R} is a set of ordered pairs of the form $\{a,b\}$ where $a \in A, b \in B$. The second element is called the \textbf{image} of the first element.\\
		
		\begin{itemize}
			\item \textbf{Domain}: The set of all first elements of the ordered pairs in a relation R from a set A to a set B is called the domain of the relation R.
			\item \textbf{Image}: The set of all second elements in a relation R from a set A to a set B is called the range of the relation R. 
			\item \textbf{Co-Domain}: The whole set B is called the codomain of the relation R.
		\end{itemize}
		
		Example, let $A = \{1,2,3,4,5,6\}$. A relation $R$ from $A$ to $A$ defined by $R = \{(x,y):y=x+1\}$ is $\{(1,2),(2,3),(3,4),(4,5),(5,6)\}$. Below is how we can depict this relation using an \textbf{Arrow Diagram}: \\
		
		\includegraphics[scale=0.6]{relation1} \\
		
		The domain is $\{1,2,3,4,5\}$, range is $\{2,3,4,5,6\}$ and the co-domain is $\{1,2,3,4,5,6\}$

		\begin{mdframed}[backgroundcolor=yellow]
			The total number of relations that can be defined from set $A$ to a set $B$ is the number of possible subsets of $A X B$. If $n(A)=p$ and $n(B)=q$, then $n(AXB)=pq$ and the total number of relations is $2^{pq}$.
		\end{mdframed}
		
		\section{Functions}
		
		A relation $f$ from set $A$ to set $B$ is said to be a function if every element of set $A$ has \textbf{one and only one} image in set $B$ and is denoted by $f: A \rightarrow B$. \\
		\\
		\textbf{Real Valued Function} is a function whose range is $R$ or a subset of $R$. \\
		\textbf{Real Function} is a function whose domain and range  is $R$ or a subset of $R$. \\

	\chapter{Trigonometry}
	\section{Trigonometric Ratios}
	 \begin{wrapfigure}{R}{0.3\textwidth}
	 	\includegraphics[scale=0.9,width=0.25\textwidth]{rat1}
	 \end{wrapfigure}
	Consider the right angle triangle ABC as shown below: \\
	 $\sin A = (Side \: opposite \:  to \: angle \: A) / Hypotnuse$ \\
	 $\cos A =(Side \: adjacent \: to \: angle \: A) / Hypotnuse$ \\
	 $\tan A = (Side \: opposite \: to \: angle \: A) / (Side \: adjacent\: to\: angle\: A)$ \\
	 $\csc A = 1 / (\sin A)$ \\
	 $\sec A = 1/(\cos A)$ \\
	 $\cot A = 1/(\tan A)$ \\
	 \section{Identities}
	 \begin{mdframed}[backgroundcolor=yellow]
		\begin{itemize}
			\item $\sin^2\theta + \cos^2\theta = 1$
			\item $\sec^2\theta - \tan^2\theta = 1$
			\item $\csc^2\theta - \cot^2\theta = 1$
			\item $\tan^2\theta - \sin^2\theta = \tan^2\theta\sin^2\theta$
			\item $\cot^2\theta - \cos^2\theta = \cot^2\theta\cos^2\theta$
			\item $\sin^4\theta + \cos^4\theta = 1-2\sin^2\theta\cos^2\theta$
			\item $\sin^6\theta + \cos^6\theta = 1 -3\sin^2\theta\cos^2\theta$
		\end{itemize}
	 \end{mdframed}
	 
	 \section{Measurement of Angles - Radian}
	 1 Radian, $1^c$ is the angle created by an arc of length $r$ in a circle of radius $r$. \\
	 $\pi^c = 180\deg$ \\
	 
	\includegraphics[scale=0.6]{radians} \\ \\ \\
	
	\begin{mdframed}[backgroundcolor=yellow]
		If in a circle of radius $r$, an arc of length $l$ subtends an angle $\theta$ radian at the centre, we have $\theta = l/r$
	\end{mdframed}
	 
	 From the below figure, we can see that:
	 \begin{itemize}
		\item $\sin\theta = (Y-Coordinate)/Radius$
		\item $\cos\theta = (X-Coordinate)/Radius$
		\item $\tan\theta = (Y-Coordinate)/(X-Coordinate)$
	 \end{itemize}
	 We can then determine if $\sin\theta, \cos\theta, \tan\theta$ will be positive or negative in each quadrant. \\ \\
	 \includegraphics[scale=0.6]{unitcircle} \\
	 
	 Below is the relation between degree measures and radian measure of some common angles: \\
	 
	 \includegraphics[scale=0.6]{raddegree}
	 
	 \section{Values of trigonometric functions for $0^{\degree}, 30^{\degree}, 45^{\degree},60^{\degree},90^{\degree}$}
	 
	 \includegraphics[scale=0.6]{values} \\
	
	\section{Reduction Formula}
	
	Consider the below figure - the two triangles are congruent (Angle-Side-Angle) and therefore the side opposite to the same angles will be equal.\\
	From the second figure, we can see that $\sin(90-A) = {x/r}$. However, we can also see that $(x/r)= \cos{A}$ from the first figure. Therefore, $\sin{(90-A)}= \cos{A}$
	\begin{mdframed}[backgroundcolor=yellow]
		\begin{itemize}
			\item$\sin(90-A) = \cos{A}$
			\item $\cos(90-A) = \sin{A}$
			
		\end{itemize}
	\end{mdframed}
	
	\includegraphics[scale=0.6]{compangle} \\
	
	Now, lets see how we can find values of $90+A$ \\
	
	\includegraphics[scale=0.6]{compangle1} \\
	
	
	We apply the same concept of congruent triangle above, and we can see that $\sin(90+A) = (Y-Coordinate \: of \: Q)/Radius = x/r$. However, $x/r = \cos{A}$, therefore $\sin(90+A)=cos{A}$. \\
	Similarly, $cos(90+A) = (X-Coordinate \: of \: Q)/Radius = (-y/r) = -\sin{A}$. \\
	
	\begin{mdframed}[backgroundcolor=yellow]
		\begin{itemize}
			\item$\sin(90+A) = \cos{A}$
			\item $\cos(90+A) = -\sin{A}$
			
		\end{itemize}
	\end{mdframed}

	Below diagram shows how we can find trigonimetric ratios of angle $180-A$
	
	\includegraphics[scale=0.6]{180theta}
	
		\begin{mdframed}[backgroundcolor=yellow]
			\begin{itemize}
				\item$\sin(180-A) = \sin{A}$
				\item $\cos(180-A) = -\cos{A}$
				
			\end{itemize}
		\end{mdframed}
		
	Below diagram shows how we can find trigonimetric ratios of angle $180+A$
	
	\includegraphics[scale=0.6]{180ptheta}
	
	\begin{mdframed}[backgroundcolor=yellow]
		\begin{itemize}
			\item$\sin(180+A) = -\sin{A}$
			\item $\cos(180+A) = -\cos{A}$
			
		\end{itemize}
	\end{mdframed}
	
	
	We can use the below figure for reduction formula 
	\begin{itemize}
		\item Along the X-Axis, there's no change (i.e. sin will remain $\sin$, $\cos$ will remain cos etc.)
		\item Along the Y-axis, $\sin$ will change to $\cos$ and vice versa, $\cot$ will change to $\tan$ and vice versa, $\csc$ will change to $\sec$ and vice versa.
		\item Along the positive X-Axis, we get even multiples of $\pi$
		\item Along the negative X-Axis, we get odd multiples of $\pi$
		\item Along the positive Y-Axis, when we divide the $\pi$ multiple by $4$, we get a remainder of $1$.
		\item Along the negative Y-Axis, when we divide the $\pi$ multiple by $4$, we get a remainder of $3$.
	\end{itemize}
	
	\includegraphics[scale=0.5]{reduction} \\
	
	Let's see the below examples on how we can use the above figure: \\
	
	\begin{itemize}
		\item $\sin(3\pi/2 - A)$: Since $3\pi/2$ is along the  negative Y-Axis, we will change $\sin$ to $\cos$. Also, since $\sin$ is negative in $3^{rd}$ Quadrant, $\sin(3\pi/2 - A) = -\cos{A}$ 
		\item $\tan(39\pi/2 - A)$:  If we divide $39$ by $4$, we get a remainder of $3$ - so $39\pi/2$ lies along the negative Y-Axis. We will first change $\tan$ to $\cot$ and since $\tan$ is positve in $3^{rd}$ quadrant, $\tan(39\pi/2 - A) = \cot{A}$. 
	\end{itemize}
	
	\includegraphics[scale=0.5]{reductionex}
	
	
	
\end{document}